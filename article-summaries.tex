\documentclass{article}

%packages included
\usepackage{amsmath}
\usepackage{graphicx}
\usepackage{tabularx}		% lets you choose width of tabular(x) environment
%\usepackage{titlesec}		% see redefinition of \section command
\usepackage{minted}		% compile with -shell-escape flag to get syntax highlighting for code blocks (\begin{minted}{<programming lang>} <...>\end{minted}, \inputminted{programming lang}{filename})
\usepackage{times}		% Times New Roman. \documentclass{article}[12] gives 12 pt. writing.
\usepackage{hyperref}
\hypersetup{
    colorlinks=true,
    linkcolor=blue,
    filecolor=magenta,      
    urlcolor=cyan,
    pdftitle={Overleaf Example},
    pdfpagemode=FullScreen,
    }

\urlstyle{same}

%custom commands:
\newcommand{\vectorarrow}{\overset{\rightharpoonup}}
%\titleformat{\section}{\normalfont\Large\bfseries}{\thesection}{1em}{}[\hrule]		% redefines \section command. if active along with package "titlesec", will add line under title of each new section

\title{Summary of articles}
\author{RoadAI group project}
\date{Created Aug. $30^{th}$ 2023}

\begin{document}

\begin{titlepage}
\maketitle
%\tableofcontents
\end{titlepage}

\subsection*{Useful links}
\begin{description}
  \item[Project description]\ref{sec.porj_desc} \url{https://www.nora.ai/competition/roadai-competition/}
  \item[MARL Wikipedia]\ref{sec.wiki} \url{\includegraphics[scale=.4]{early-MARL-algorithms}}
  \item[Multiagent Reinforcement Learning PowerPoint, DeepMind]\ref{sec.deepmind} \url{https://rlss.inria.fr/files/2019/07/RLSS_Multiagent.pdf}
  \item[StarCraft Multi-Agent Challenge]\ref{sec.smac} \url{\includegraphics[scale=.4]{early-MARL-algorithms}}
  \item[Apple Core Motion documentation] \url{https://developer.apple.com/documentation/coremotion/getting_raw_accelerometer_events}
\end{description}


\section*{Project Description}          \label{sec.proj_desc}
\mintinline{python}{pandas.read_hdf} to read \texttt{hdf} vibration files.

\noindent
\textbf{Reasonable assumptions to make:}
\begin{itemize}
  \item Excavators have infinite mass to move.\\
  \item Long term plan isn't digitally available.\\
  \item Noone will be using the system if they have to add a bunch of extra info. manually.
\end{itemize}


\section*{Wikipedia}          \label{sec.wiki}


\section*{DeepMind presentation, MARL}\label{sec.deepmind}
\includegraphics[scale=.4]{algorithm-value-iteration}
\includegraphics[scale=.4]{algorithm-minimax-q}
\includegraphics[scale=.4]{early-MARL-algorithms}
\includegraphics[scale=.4]{early-MARL-algorithms-2}
\includegraphics{algorithm-exploratory-descent}


\section*{StarCraft Multi-Agent Challenge}        \label{sec.smac}
\mintinline{python}{PyMARL} is an open-source learning framework based on
\mintinline{python}{PyTorch} which serveds as a template for dealing with deep MARL algorithms.




\end{document}

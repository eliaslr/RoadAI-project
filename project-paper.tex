\documentclass[conference]{IEEEtran}
\IEEEoverridecommandlockouts
% The preceding line is only needed to identify funding in the first footnote. If that is unneeded, please comment it out.
\usepackage{cite}
\usepackage{amsmath,amssymb,amsfonts}
\usepackage{algorithmic}
\usepackage{graphicx}
\usepackage{textcomp}
\usepackage{xcolor}
\def\BibTeX{{\rm B\kern-.05em{\sc i\kern-.025em b}\kern-.08em
    T\kern-.1667em\lower.7ex\hbox{E}\kern-.125emX}}

\begin{document}




% -----------------------------------
% -------------- TITLE --------------
\title{RoadAI - Reducing Emissions in Road Construcrtion
\thanks{Identify applicable funding agency here. If none, delete this.}
}
% -----------------------------------


% -----------------------------------
% ------------- AUTHORS -------------
\author{
  \IEEEauthorblockN{1\textsuperscript{st} Given Name Surname}
  \IEEEauthorblockA{\textit{Dept. of Informatics} \\
  \textit{University of Oslo}\\
  Oslo, Norway \\
  email}
  \and
  \IEEEauthorblockN{2\textsuperscript{nd} Given Name Surname}
  \IEEEauthorblockA{\textit{Dept. of Informatics} \\
  \textit{University of Oslo}\\
  Oslo, Norway \\
  email}
  \and
  \IEEEauthorblockN{3\textsuperscript{rd} Given Name Surname}
  \IEEEauthorblockA{\textit{Dept. of Informatics} \\
  \textit{University of Oslo}\\
  Oslo, Norway \\
  email}
  \and
  \IEEEauthorblockN{4\textsuperscript{th} Given Name Surname}
  \IEEEauthorblockA{\textit{Dept. og Informatics} \\
  \textit{University of Oslo}\\
  Oslo, Norway \\
  email}
  \and
  \IEEEauthorblockN{5\textsuperscript{th} Given Name Surname}
  \IEEEauthorblockA{\textit{Dept. og Informatics} \\
  \textit{University of Oslo}\\
  Oslo, Norway \\
  email}
}

\maketitle


% -----------------------------------
% ------------ ABSTRACT -------------
% \begin{abstract}
% % Abstract is last thing we should write.

% % Should have 3-5 keywords.
% Index Terms - Multi-Agent Reinforcement Learning (MARL), 
% \end{abstract}


% -----------------------------------
% - INTRODUCTION (and MOTIVATION?)  -
\section{Introduction}
In Norway, the total CO2 emissions from construction machines is 1.5\%. \cite{noraRoadAIReducing}
This paper aims to reduce emissions from road construction using Multi-Agent Reinforcement Learning
(MARL) techniques.

The project is based on the RoadAI competition by Norwegian Artificial Intelligence Research
Consortium (NORA) \cite{noraRoadAIReducing}. Participants in the competitition get access to four
different sets of data. The provided data includes GPS data recorded from iPads in dump trucks and
trucks, detailing timestamps, machine IDs, locations, material type and quantity, and loading/unloading
locations. This data is organized into trips. Additional metadata about the trips is also available.
Machine data, known as AEMP, offers information on Skanska-owned machines, including location,
odometer readings, fuel consumption, and usage hours. Vibration data, though duplicated,
presents an opportunity for analysis, with records occurring seven times. This data, collected from
iPads, includes three-dimensional vibration data. Drone data, generated using ArcGIS Site Scan software,
comprises ortho-mosaic images, point clouds, mesh data, digital terrain models, and digital surface models,
potentially useful for automated progress reports.

Our main objective is twofold: (1) reducing emissions by optimizing idle time of dumptrucks, and (2)
optimizing paths which reduce CO2 emissions as much as possible. Objective (2) can be achieved through
planning paths which reduce acceleration, for example by taking topography into account.

This paper will present an implementation of a customized PettingZoo\footnote{https://pettingzoo.farama.org}
environment.


% -----------------------------------
% ------------- METHODS -------------
\section{Methodology}


% -----------------------------------
% ---------- RELATED WORK -----------
\section{Related works}
SMACv2 \cite{ellis2022smacv2} evaluates two MARL algorithms for problems using centralized training with
decentralized execution (CTDE); QMIX and MAPPO. QMIX is a multi-agent adaptation of the Deep Q-Network method.
MAPPO is a multi-agent adaptation of the Proximal Policy Optimization method. On the SMACv2 testbed, QMIX
had the best performance out of the two, but MAPPO 


% -----------------------------------
% ----------- REFERENCES ------------
% \newpage
% \nocite{*}        % This includes all references that haven't been explicitly cited
\bibliography{citations.bib}{}
\bibliographystyle{plain}

\end{document}
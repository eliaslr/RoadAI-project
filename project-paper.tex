\documentclass[conference]{IEEEtran}
\IEEEoverridecommandlockouts
% The preceding line is only needed to identify funding in the first footnote. If that is unneeded, please comment it out.
\usepackage{cite}
\usepackage{amsmath,amssymb,amsfonts}
\usepackage{algorithmic}
\usepackage{graphicx}
\usepackage{textcomp}
\usepackage{xcolor}
\def\BibTeX{{\rm B\kern-.05em{\sc i\kern-.025em b}\kern-.08em
    T\kern-.1667em\lower.7ex\hbox{E}\kern-.125emX}}

\begin{document}




% -----------------------------------
% -------------- TITLE --------------
\title{RoadAI - Reducing Emissions in Road Construcrtion
\thanks{Identify applicable funding agency here. If none, delete this.}
}
% -----------------------------------


% -----------------------------------
% ------------- AUTHORS -------------
\author{
  \IEEEauthorblockN{1\textsuperscript{st} Given Name Surname}
  \IEEEauthorblockA{\textit{Dept. of Informatics} \\
  \textit{University of Oslo}\\
  Oslo, Norway \\
  email}
  \and
  \IEEEauthorblockN{2\textsuperscript{nd} Given Name Surname}
  \IEEEauthorblockA{\textit{Dept. of Informatics} \\
  \textit{University of Oslo}\\
  Oslo, Norway \\
  email}
  \and
  \IEEEauthorblockN{3\textsuperscript{rd} Given Name Surname}
  \IEEEauthorblockA{\textit{Dept. of Informatics} \\
  \textit{University of Oslo}\\
  Oslo, Norway \\
  email}
  \and
  \IEEEauthorblockN{4\textsuperscript{th} Given Name Surname}
  \IEEEauthorblockA{\textit{Dept. og Informatics} \\
  \textit{University of Oslo}\\
  Oslo, Norway \\
  email}
  \and
  \IEEEauthorblockN{5\textsuperscript{th} Given Name Surname}
  \IEEEauthorblockA{\textit{Dept. og Informatics} \\
  \textit{University of Oslo}\\
  Oslo, Norway \\
  email}
}

\maketitle


% -----------------------------------
% ------------ ABSTRACT -------------
\begin{abstract}
% Abstract is last thing we should write.

% Should have 3-5 keywords.
Index Terms - Multi-Agent Reinforcement Learning (MARL), 
\end{abstract}


% -----------------------------------
% ---------- INTRODUCTION -----------
\section{Introduction}
The construction industry plays a pivotal role in global economic development,
providing essential infrastructure and buildings to support growing populations
and urbanization. However, this sector is also a significant contributor to
environmental challenges, particularly in terms of greenhouse gas emissions.

Finding innovative solutions to reduce the carbon footprint of construction
activities has become an imperative for sustainability and environmental stewardship.

\noindent
This paper aims to use Multi-Agent Reinforcement Learning in order to reduce emissions.
This will be done through optimization of idle time as well as minimization of
acceleration. 


% -----------------------------------
% ----------- REFERENCES ------------
% \newpage
% \nocite{*}        % This includes all references that haven't been explicitly cited
\bibliography{citations.bib}{}
\bibliographystyle{plain}

\end{document}
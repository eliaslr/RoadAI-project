\documentclass[conference]{IEEEtran}
\IEEEoverridecommandlockouts
% The preceding line is only needed to identify funding in the first footnote. If that is unneeded, please comment it out.
\usepackage{cite}
\usepackage{amsmath,amssymb,amsfonts}
\usepackage{graphicx}
\usepackage{textcomp}
\usepackage{url}
\usepackage{xcolor}
\def\BibTeX{{\rm B\kern-.05em{\sc i\kern-.025em b}\kern-.08em
    T\kern-.1667em\lower.7ex\hbox{E}\kern-.125emX}}
\begin{document}

% Custom commands:
  \newcommand{\coo}{\ensuremath{\mathrm{CO_2}}}

% -----------------------------------
% -------------- TITLE --------------
\title{RoadAI - A Multiagent Reinforcement Learning Approach to Reducing \coo Emissions at a Construction Site}
% -----------------------------------


% -----------------------------------
% ------------- AUTHORS -------------
\author{
  \IEEEauthorblockN{Viktor Ringvold Hasle}
  \IEEEauthorblockA{\textit{Dept. of Informatics} \\
  \textit{University of Oslo}\\
  Oslo, Norway \\
  viktorrh@ifi.uio.no}
  \and
  \IEEEauthorblockN{Ada Hatland}
  \IEEEauthorblockA{\textit{Dept. of Informatics} \\
  \textit{University of Oslo}\\
  Oslo, Norway \\
  adaha@ifi.uio.no
  }
  \and
  \IEEEauthorblockN{Elias Lynum Ringkjøb}
  \IEEEauthorblockA{\textit{Dept. of Informatics} \\
  \textit{University of Oslo}\\
  Oslo, Norway \\
  eliaslr@ifi.uio.no
 }
  \and
  \IEEEauthorblockN{Ilya Berezin}
  \IEEEauthorblockA{\textit{Dept. of Informatics} \\
  \textit{University of Oslo}\\
  Oslo, Norway \\
  ilyab@ifi.uio.no}
  \and \\
  \IEEEauthorblockN{Tom Frode Hansen}
  \IEEEauthorblockA{\textit{NGI- Norges Geotekniske Institutt} \\
  Oslo, Norway \\
  tom.frode.hansen@ngi.no
  }
}

\maketitle


% -----------------------------------
% ------------ ABSTRACT -------------
% -----------------------------------
% \begin{abstract}
% % Abstract is last thing we should write.

% % Should have 3-5 keywords.
% Index Terms - Multi-Agent Reinforcement Learning (MARL), 
% \end{abstract}


% -----------------------------------
% - INTRODUCTION  -
% -----------------------------------
\section{Introduction}
Carbon dioxide \coo emission is one of the largest contributions to greenhouse gasses and global warming.
Reducing \coo emissions is an important step in preventing global warming and all of the negative
consequences associated with it. Norway's total greenhouse gas emissions were, according to Statistics
Norway, in 2022 48.9 million tonnes of \coo equivalences. <Insert reference> 1.5\% of which are emitted
by construction machines. \cite{noraRoadAIReducing} The reduction of such a significant percentage of
the total emissions is a crucial task.

This project is based on the RoadAI competition held by Norwegian Artificial Intelligence Research
Consortium (NORA). \cite{noraRoadAIReducing} The objective in the RoadAI competition is to reduce the
total \coo emissions from a road construction site using machine learning methods.In this paper, we will
present a method for using Multiagent Reinforcement Learning (MARL) algorithms in order to reduce such
emissions.

At the construction site, there will be a designated area for the construction of a road. There are two
different types of agents: dump trucks and excavators. The excavators fill dump trucks with <road mass??>,
which the dump trucks then carry away to leave at the site of the unfinished road.

% ::::Discard?::::
% Our main objective is twofold: (1) reducing emissions by optimizing idle time of dump trucks and (2)
% optimizing paths which reduce \coo emissions as much as possible. Objective (2) can be achieved through
% planning paths which reduce acceleration, for example by taking topography into account.

% This paper will present an implementation of a customized PettingZoo\footnote{https://pettingzoo.farama.org}
% environment.


% -----------------------------------
% ------------- METHODS -------------
% -----------------------------------
\section{Methodology}

The goal of the Road AI project is to demonstrate an algorithm that could reduce emissions from road construction.
We are using MARL to train agents on a simulation of road construction and to demonstrate how it can be utilized to solve this problem.
Our methodology primarily focuses on the utilization of Multi-Agent Reinforcement Learning to tackle the problem of emissions in road construction. Given the complexity of the task, we adopted a step-by-step approach to ensure that the model is not only accurate but also efficient.

\subsection{Problem Formulation}
The first step in our methodology was to formulate the problem in the context of MARL. We defined each construction vehicle as an agent with its own set of actions, observations, and rewards. The joint action space consisted of moving, idling, loading, or unloading materials. Observations for each agent included the status of the road construction, the position of other agents, and the current load of materials. Rewards were also implemented to promote actions that lead to reduced emissions.

\subsection{Simulation Environment}
To train our agents, we developed a custom simulation environment using the PettingZoo framework. The environment replicates a typical road construction site, with details like topography and material locations. In this environment every object is represented as a particle in a grid. 
The agents of the simulation, construction trucks are able to move cardinally in this grid with the goal of hauling construction materials from excavators to the unfinished road. 
Every agent gets assigned a reward for each action they take with positive rewards for productive moves, and negative rewards for unproductive ones (idling, or unnecessary moving). 

While simulating the trucks we collect the rewards which the reinforcement learning algorithms uses to train.
Using the average rewards we can estimate the algorithms performance. After simulating 10 episodes of 10,000 time steps we have a firm grasp over if an algorithm performed well.

\subsection{Training Procedure}
Training conducted using state-of-the-art MARL algorithms, namely MAPPO and QMIX. Each agent is initialized with random policies, and as the training progressed, they learn how to collaborate and make decisions that minimized the overall emissions.

\begin{itemize}
\item \textbf{MAPPO}: A decentralized actor-critic method that leverages a centralized value function to enhance training stability.
\item \textbf{QMIX}: A value-based method that uses a centralized critic to estimate joint action-value functions.
\end{itemize}

\subsection{Evaluation}
Post-training, we evaluated our models based on following metrics:
\begin{enumerate}
\item ....TBA
\end{enumerate}

\subsection{Hyperparameter Optimization}
To ensure the best performance of our models, we used the Hydra framework\cite{Yadan2019Hydra} to manage and optimize the system's hyperparameters. Using a grid search approach, we explore various configurations and selecte the one that provides with the best results.


% -----------------------------------
% ------ RESULTS AND DISCUSSION -----
% -----------------------------------


% -----------------------------------
% ---------- RELATED WORK -----------
% -----------------------------------
\section{Related works}
SMACv2\cite{ellis2022smacv2} evaluates two MARL algorithms for problems using centralized training with
decentralized execution (CTDE); QMIX and MAPPO.

QMIX employs a centralized critic to estimate joint action-value functions, while MAPPO uses a decentralized
actor-critic framework with a centralized value function.

QMIX outperformed MAPPO on most scenarios, but QMIX is memory-intensive due to its large replay buffer. For
this reason, QMIX requires more computational power. In addition, the paper notes that MAPPO was still
increasing its performance at the time of termination in several scenarios, which indicates it could end up
with a better result.

This paper aims to evaluate both the QMIX and MAPPO algorithms.


Flatlands\cite{laurent2021flatland} was a competition where contestants to design a MARL algorithm for 
train scheduling. This is similar to the approach we seek to use to solve RoadAI. 
In the competition MAPPO performed well and is a pointer for what we could use in our own MARL implementation.


The application of Multi-Agent Reinforcement Learning (MARL) for carbon reduction in the construction industry is relatively recent field. Here are some other related works:

\begin{itemize}

\item \textbf{Optimized Resource Allocation:}
MARL has been used to optimize resource allocation in construction sites, ensuring that machinery and equipment are used in a most efficient way. By minimizing idle times and optimizing machinery routes, fuel consumption can be reduced, leading to a decrease in carbon emissions\cite{resource_allocation}.

\item \textbf{Dynamic Scheduling:}
Construction projects often involve multiple tasks that need a proper coordination. MARL algorithms has been used for dynamic scheduling, ensuring that tasks are carried out in an optimized way, reducing potential delays and minimizing the carbon footprint due to inefficiencies\cite{dynamic_scheduling}.

\end{itemize}

While the above works have shown the potential of MARL, there is still potential to explore with the advent of newer algorithms. The main challenge remains in integrating these advanced techniques seamlessly into the traditional construction processes and ensuring real-world applicability.

\noindent


% -----------------------------------
% ----------- CONCLUSION ------------
% -----------------------------------
\section{Conclusion}




% -----------------------------------
% ----------- REFERENCES ------------
% -----------------------------------
\newpage
% \nocite{*}        % This includes all references that haven't been explicitly cited
\bibliography{citations.bib}
\bibliographystyle{plain}


% -----------------------------------
% -------- ETHICS STATEMENT ---------
% -----------------------------------
\newpage

\appendices 
\section{Ethics Statement}
In the course of our project on Multiagent Reinforcement Learning, aimed at mitigating \coo emissions
from construction sites, we have conducted research that involves no experiments with human participants.
While this specific phase of our work does not directly raise concerns regarding wages or human
participation, it is essential to recognize that the technologies and methodologies we are developing
may eventually be applied in real-world settings where human interactions are integral. In such instances,
it becomes crucial to address ethical considerations associated with human safety and well-being.

We acknowledge that our current implementation serves as a simulation, and it is not as robust as a
real-world application would need to be. However, we are committed to exploring ways to ensure that the
agents, which are part of the system, incorporate safeguards to prevent harm to humans and machinery.
This commitment reflects our ethical responsibility to uphold the safety and welfare of individuals who
may interact with these agents in practical applications.

Moreover, our reinforcement learning framework includes a reward function that assigns a substantial
negative reward for agent collisions, with the intent of discouraging such incidents. The severity of
this negative feedback is designed to minimize the likelihood of agents crashing into one another. It
is important to note, though, that while our design minimizes such events, it does not guarantee their
complete avoidance.

The dataset we have employed for this project, generated by Skanska Norge AS and provided by Norwegian
Artificial Intelligence Research Consortium (NORA), does not contain any personally identifiable
information. Although the dataset's availability has yet to be confirmed, it was shared with us by NORA
following the conclusion of the RoadAI competition.

The primary objective of our project is to significantly reduce \coo emissions originating from
construction sites, potentially leading to substantial environmental and societal benefits.
Construction sites contribute 1.5\% of total Norwegian \coo emissions, making the reduction of
emissions in this sector a vital endeavour. Implementing our system across numerous construction sites
in Norway could result in a significant reduction in total \coo emissions, a crucial step toward a more
sustainable future.

In line with our commitment to ethical AI deployment, transparency and explainability need to be
addressed. It is imperative to ensure stakeholders can understand and trust the model's decisions.
Thus we have to make efforts to ensure that MARL algorithms are interpretable, allowing a comprehensive
understanding of factors influencing decisions.

Our research and project implementation do not support or enable any form of discrimination against
individuals or groups. Furthermore, they do not entail risks related to deception or harassment.
Our work is firmly rooted in legal activities and does not promote any restrictions on human rights.
Our ethical commitment is to conduct research and develop technologies that contribute to a greener,
more sustainable world without compromising the well-being of individuals or society as a whole.


\end{document}

% STRUCTURE:
% ----------

% Title - 1 sentence
%   (Your paper title should be specific, concise, and descriptive. Avoid using unnecessary words such as “new” or “novel”. Include keywords that will help a reader find your paper.)
%   Proposal: RoadAI - A Multiagent Reinforcement Learning approach to reducing \coo emissions at a construction site

% Abstract - 4 sentences
%   (Provide a concise summary of the research conducted. Include the conclusions reached and the potential implications of those conclusions. Your abstract should also:
%   - consist of a single paragraph up to 250 words, with correct grammar and unambiguous terminology;
%   - be self-contained with no abbreviations, footnotes, references, or mathematical equations;
%   - highlight what is unique in your work;
%   - include 3-5 keywords or phrases that describe the research, with any abbreviations clearly defined,  to help readers find your paper.)

% Introduction - 0.5-1 pages
%   (Help the reader understand why your research is important and what it is contributing to the field.
%   - Start by giving the reader a brief overview of the current state of research in your subject area.
%   - Progress to more detailed information on the specific topic of your research.
%   - End with a description of the exact question or hypothesis that your paper will address.
%   Also state your motivation for doing your research and what it will contribute to the field.)

% Methods - 2.5-3 pages
%   (Formulate your research question. It should include:
%   - a detailed description of the question;
%   - the methods you used to address the question;
%   - the definitions of any relevant terminology;
%   - any equations that contributed to your work.
%   The methods section should be described in enough detail for someone to replicate your work.)

%   Description of research question

%   Environment

%   PPO

%   DQN

% Results and Discussion - 0.5-1 page
%   (Show the results that you achieved in your work and offer an interpretation of those results. Acknowledge any limitations of your work and avoid exaggerating the importance of the results.)

% Related Work - 1.5-2 pages
%   (Related work)

% Conclusion - 1 page
%   (Summarize your key findings. Include important conclusions that can be drawn and further implications for the field.)

%   Future work
%     (Discuss benefits or shortcomings of your work and suggest future areas for research.)

% Acknowledgements - ????
%   (You can recognize individuals who provided assistance with your work, but who do not meet the definition of authorship. The acknowledgments section is optional.)

% References
